\documentclass[aspectratio=169]{beamer}

% 使用自定义简约主题(母版)
\usepackage{mybasic}

% 主题元信息(根据需要修改成自己的内容)
\title{示例演示 / Demo Presentation}
\subtitle{简约 beamer 母版(中英双语、封面/目录/分节/致谢)}
\author[Lin]{Lin Ren / 任林}
\newcommand{\advisor}{指导老师:肖国辉}
\institute{Southeast University \\ Institute of Cognitive Intelligence (COIN)}
\date{\today}
% 建议在页脚中放置实验室信息
\footlinecenter{东南大学 COIN 模板}

% 只保留 Outline,不再自动插入 Part.X 分节页
\DisableSectionPage

% 目录显示控制
\setcounter{tocdepth}{2}

\begin{document}

% 首页(封面)
\begin{frame}[plain]
  \titlepage
\end{frame}

% 总目录页(展示所有章节)
\begin{frame}{Outline / 总目录}
  \tableofcontents
\end{frame}

% 功能索引页:点击条目可跳转到对应示例页
\begin{frame}{Feature Index / 功能索引}
  \begin{columns}[t]
    \column{0.48\textwidth}
      \textbf{基础示例}
      \begin{itemize}
        \item \hyperlink{frame-how-to-use}{快速上手:如何使用本模板}
        \item \hyperlink{frame-minimal-example}{最小示例:最小 Beamer 代码}
        \item \hyperlink{frame-plain-text}{文本与中英混排示例}
        \item \hyperlink{frame-lists}{项目符号与编号列表}
      \end{itemize}

    \column{0.48\textwidth}
      \textbf{高级示例}
      \begin{itemize}
        \item \hyperlink{frame-math-inline}{数学公式:行内与行间}
        \item \hyperlink{frame-math-aligned}{数学公式:对齐公式}
        \item \hyperlink{frame-figures}{插入图片示例}
        \item \hyperlink{frame-table}{简单表格示例}
        \item \hyperlink{frame-examplebox}{示例框 examplebox}
        \item \hyperlink{frame-highlight-1}{高亮文本块 highlightblock}
        \item \hyperlink{frame-tikz-diagram}{TikZ 示意图}
        \item \hyperlink{frame-thanks}{致谢页模板}
      \end{itemize}
  \end{columns}
\end{frame}

% ---------------- Sections 内容从 ./sections 引入 ----------------

\section{Quick Start / 快速上手}

\begin{frame}[label=frame-how-to-use]{如何使用本模板 / How to Use}
  \begin{enumerate}
    \item 使用 XeLaTeX 编译 \texttt{main.tex};
    \item 修改文首的 \texttt{\textbackslash title}、\texttt{\textbackslash author}、\texttt{\textbackslash date} 等元信息;
    \item 使用 \texttt{\textbackslash section} 和 \texttt{\textbackslash subsection} 组织内容;
    \item 参考后续章节中的“经典示例”,复制相应代码并替换为自己的内容。
  \end{enumerate}
\end{frame}

\begin{frame}[fragile,label=frame-minimal-example]{最小示例 / Minimal Example}
% 提示:在 beamer 的 fragile frame 里,尽量避免在 verbatim 中出现 \begin{frame} / \end{frame}
\begin{verbatim}
\section{Introduction}

\begin{itemize}
  \item Hello, Beamer!
\end{itemize}
\end{verbatim}
\end{frame}



\section{Text \& Lists / 文本与列表}

\begin{frame}[label=frame-plain-text]{普通文本 / Plain Text}
  本模板默认采用 Arial + 宋体(SimSun),适合中英混排:
  \begin{itemize}
    \item 英文示例:This is a simple English sentence.
    \item 中文示例:这是一段用于演示的中文文本。
    \item 中英混排:Beamer 模板支持 English + 中文 混合排版。
  \end{itemize}
\end{frame}

\begin{frame}[label=frame-lists]{项目符号与编号 / Itemize \& Enumerate}
  \begin{columns}[t]
    \column{0.48\textwidth}
      \textbf{无序列表 / Itemize}
      \begin{itemize}
        \item Point A
        \item Point B
        \item Point C
      \end{itemize}
    \column{0.48\textwidth}
      \textbf{有序列表 / Enumerate}
      \begin{enumerate}
        \item 第一步
        \item 第二步
        \item 第三步
      \end{enumerate}
  \end{columns}
\end{frame}



\section{Math / 数学公式}

\begin{frame}[label=frame-math-inline]{行内与行间公式 / Inline \& Display}
  行内公式示例:令 \( f(x) = ax^2 + bx + c \)。\\[0.5em]
  行间公式示例:
  \[
    \int_a^b f(x)\,\mathrm{d}x = F(b) - F(a)
  \]
\end{frame}

\begin{frame}[label=frame-math-aligned]{对齐公式 / Aligned Equations}
  以二项式定理为例:
  \[
    (a + b)^n = \sum_{k=0}^{n} \binom{n}{k} a^{n-k} b^k
  \]
  以及一组对齐公式:
  \[
    \begin{aligned}
      E[X] &= \sum_{i} x_i p_i,\\
      \mathrm{Var}(X) &= E[X^2] - (E[X])^2.
    \end{aligned}
  \]
\end{frame}



\section{Figures \& Tables / 图片与表格}

\begin{frame}[label=frame-figures]{插入图片 / Figures}
  \begin{figure}
    \centering
    \includegraphics[width=0.32\textwidth]{images/template/seu_logo_yello_and_green}
    \caption{东南大学校徽示例 / SEU Logo}
  \end{figure}
\end{frame}

\begin{frame}[label=frame-table]{简单表格 / Simple Table}
  \begin{table}
    \centering
    \begin{tabular}{lcc}
      \hline
      指标 & 值1 & 值2 \\
      \hline
      Accuracy & 95\% & 97\% \\
      F1-Score & 0.90 & 0.93 \\
      \hline
    \end{tabular}
    \caption{性能对比示例 / Example Performance Comparison}
  \end{table}
\end{frame}



\section{Advanced Examples / 高级示例}

\begin{frame}[label=frame-examplebox]{示例框 environment / Example Box}
  \begin{examplebox}[Key Idea / 核心思想]
    使用 \texttt{examplebox} 环境可以突出关键信息:
    \begin{itemize}
      \item 适合写“定理 / 结论 / 提示 / 注意”这类内容;
      \item 左侧主色竖条 + 轻微阴影,整体更接近“卡片风”样式;
      \item 可以嵌套普通列表、公式等内容。
    \end{itemize}
  \end{examplebox}
\end{frame}

\begin{frame}[label=frame-examplebox-more-1]{更多示例框风格 / More Example Box Styles (1)}
  \begin{examplebox}[Default / 默认示例框]
    用于一般性的“结论 / 关键步骤 / 小结”,视觉重点适中,不会喧宾夺主。
  \end{examplebox}

  \begin{exampleboxAccent}[Tip / 提示框]
    适合写“提示 / 注意 / 常见错误”:
    \begin{itemize}
      \item 使用东南大学黄色作为主色,更醒目;
      \item 适合在复杂公式或代码前后给出自然语言提醒。
    \end{itemize}
  \end{exampleboxAccent}

\end{frame}

\begin{frame}[label=frame-examplebox-more-2]{更多示例框风格 / More Example Box Styles (2)}
  \begin{exampleboxOutline}[Sketch / 极简框]
    极简细描边 + 无阴影:\par
    \begin{itemize}
      \item 适合“证明思路 / 草稿推导 / 伪代码框架”等内容;
      \item 与周围文本融为一体,只做轻微分组,不抢视线。
    \end{itemize}
  \end{exampleboxOutline}
\end{frame}

\begin{frame}[label=frame-highlight-1]{高亮文本块示例 / Highlighted Text Blocks (1)}
  \begin{highlightblockPrimary}[Block Title / 主色块标题]
    Lorem ipsum dolor sit amet, consectetur adipiscing elit. Integer lectus nisl, ultricies in feugiat rutrum, porttitor sit amet augue.
  \end{highlightblockPrimary}

  \vspace{0.35em}

  \begin{highlightblockSuccess}[Example Block Title / 成功示例块]
    Aliquam ut tortor mauris. Sed volutpat ante purus, quis accumsan justo cursus in.
  \end{highlightblockSuccess}

  \vspace{0.4em}
\end{frame}

\begin{frame}[label=frame-highlight-2]{高亮文本块示例 / Highlighted Text Blocks (2)}
  \begin{highlightblockAlert}[Alert Block Title / 警告示例块]
    Pellentesque sed tellus purus. Class aptent taciti sociosqu ad litora torquent per conubia nostra, per inceptos himenaeos.
  \end{highlightblockAlert}
\end{frame}

\begin{frame}[label=frame-multicol]{Multiple Columns}
  \framesubtitle{Subtitle}
  \begin{columns}[T,onlytextwidth]
    \column{0.42\textwidth}
      {\bfseries Heading}\par
      \vspace{0.4em}
      \begin{enumerate}
        \item Statement
        \item Explanation
        \item Example
      \end{enumerate}

    \column{0.58\textwidth}
      \begin{highlightblockPrimary}[]
        Lorem ipsum dolor sit amet, consectetur adipiscing elit. Integer lectus nisl, ultricies in feugiat rutrum, porttitor sit amet augue. Aliquam ut tortor mauris. Sed volutpat ante purus, quis accumsan dolor.
      \end{highlightblockPrimary}
  \end{columns}
\end{frame}

\begin{frame}[label=frame-tikz-diagram]{TikZ 示意图 / TikZ Diagram}
  \begin{center}
    \begin{tikzpicture}[node distance=1.8cm,>=stealth]
      \node[draw,rounded corners,fill=SEUYellow!20] (input) {Input};
      \node[draw,rounded corners,right of=input,fill=Primary!10] (model) {Model};
      \node[draw,rounded corners,right of=model,fill=SEUYellow!20] (output) {Output};
      \draw[->,thick,Primary] (input) -- (model);
      \draw[->,thick,Primary] (model) -- (output);
    \end{tikzpicture}
  \end{center}
\end{frame}



\section{Thanks / 致谢}

% 致谢页(会自动显示作者、指导老师与时间)
\thanksframe[Thank You / 谢谢]




\end{document}


